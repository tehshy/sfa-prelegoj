\documentclass{beamer}
\usepackage[T1]{fontenc}
\usepackage[utf8x]{inputenc} 
\usepackage[esperanto,polish]{babel}
%\usepackage{umbcboxes}
\usepackage{listings}

\title[Esperanto for programming]{The advantages of using constructed agglutinative language for programming}
%\subtitle{}
\author{Tomasz Szymula}
\institute[PEJ]{Polish Esperanto-Youth}

\date[2014]{AGH, English class,\\ \today}
\subject{Linguistics}

\usetheme{Antibes}
\usecolortheme{dolphin}

\AtBeginSubsection[]
{
  \begin{frame}
    \frametitle{Plan}
    \tableofcontents[currentsubsection]
  \end{frame}
}
% "define" Scala
\lstdefinelanguage{Scala}{
  morekeywords={abstract,case,catch,class,def,%
    do,else,extends,false,final,finally,%
    for,if,implicit,import,match,mixin,%
    new,null,object,override,package,%
    private,protected,requires,return,sealed,%
    super,this,throw,trait,true,try,%
    type,val,var,while,with,yield},
  otherkeywords={=>,<-,<\%,<:,>:,\#,@},
  sensitive=true,
  morecomment=[l]{//},
  morecomment=[n]{/*}{*/},
  morestring=[b]",
  morestring=[b]',
  morestring=[b]"""
}

\lstset{language=Scala} 

\begin{document}
  \frame{\titlepage}
 
  \section{Esperanto}
  \subsection{What's this} 
  
  \begin{frame}
  	\frametitle{Esperanto}
  	
  	\begin{block}{Esperanto is a language:}
  		\begin{itemize}
  			\item constructed,
			\item international,
			\item auxiliary.
		\end{itemize}
  	\end{block}
  	
  	\pause

\definecolor{antiquefuchsia}{rgb}{0.57, 0.36, 0.51}
\definecolor{ao(english)}{rgb}{0.0, 0.5, 0.0}
\definecolor{blue(pigment)}{rgb}{0.2, 0.2, 0.6}
	\begin{block}{And for linguists its agglutinative}
		Eg., I like reading\dots \textbf{\textcolor{antiquefuchsia}{libr}o\textcolor{ao(english)}{j}\textcolor{blue(pigment)}{n}}
	\end{block}  	
  	
  \end{frame}
 
  
  
  \subsection{What it's like}
  
%  \begin{frame}
%  	\frametitle{Rdzenie}
%  	\framesubtitle{Za broszurą ,,Malkovru Esperanton''}
%  	
%  	\begin{block}{Słowotwórstwo}
%  		Oparte o system prefiksów i sufiksów
%  	\end{block}
%	Rdzenie wyrazów pochodzą z języków:
%	\begin{itemize}
%		\item ~75\% z romańskich,
%		\item ~20\% z germańskich,
%		\item ~5\% z innych.
%	\end{itemize}
%		
%  \end{frame}
  
%**************************************** IL 
  \begin{frame}
  	\frametitle{Affixes}
  	\framesubtitle{-il}
  	  	
  		\begin{block}{-il}
  			\begin{center}
  				a tool
  			\end{center}
  		\end{block}
  	
  		\begin{itemize}
  			\item<1-> tranĉi -- to cut $\rightarrow$ tranĉilo -- knife
  			\item<1-> kompili -- to compile $\rightarrow$ kompililo -- compiler
  			\item<1-> ligi -- to link $\rightarrow$ ligilo -- link
  			\item<2-> manĝi -- to eat $\rightarrow$ \pause manĝilo -- a piece of cutlery
  		\end{itemize}
  \end{frame}

%**************************************** -AR
  \begin{frame}
  	\frametitle{Affixes}
  	\framesubtitle{-ar}
  	  	
  		\begin{block}{-ar}
  			\begin{center}
  				set, collection
  			\end{center}
  		\end{block}
  	
  		\begin{itemize}
  			\item<1-> arbo -- tree $\rightarrow$ arbaro -- wood
  			\item<1-> homo -- human$\rightarrow$ homaro -- humanity
  			\item<1-> klavo -- button $\rightarrow$ klavaro -- keyboard
  			\item<2-> manĝilo - a piece of cutlery $\rightarrow$ \pause manĝilaro - cutlery
  		\end{itemize}
  \end{frame}

%**************************************** -ER
  \begin{frame}
  	\frametitle{Affixes}
  	\framesubtitle{-er}
  	  	
  		\begin{block}{-er}
  			\begin{center}
  			part
  			\end{center}
  		\end{block}
  	
  		\begin{itemize}
			\item<1-> programo -- programme $\rightarrow$ programero -- part of the programme
			\item<1-> blogo -- blog $\rightarrow$ blogero -- blog note
			
  		\end{itemize}
  \end{frame}

%  \begin{frame}
%  	\frametitle{Ogólny przykład do afiksów}
%  	
%  	coś w stylu:\\
%  	manĝi - jesc,\\
%  	manĝilo - sztuciec\\
%  	manĝilaro - zastawa\\
%  	manĝeti - podjadac\\
%  	manĝegi - "wcinac"\\
%  	antaŭmanĝi - przedjadac (?)\\
%  	itd.
%
%  \end{frame}
 
  \section{W praktyce}
  \subsection{Wzorce projektowe}
  
\begin{frame}
	\frametitle{Wybrane wzorce projektowe}
	
	\begin{block}{}
		Observer, Observable $\rightarrow$ Spektanto, Spektebla
	\end{block}
	
	\begin{block}{}
		Decorator $\rightarrow$ Dekoranto
	\end{block}
	
	\begin{block}{}
		Interpreter $\rightarrow$ Interpretilo
	\end{block}
	
	\dots
\end{frame}  

\begin{frame}
	\frametitle{Wejście/wyjście}
	
	\begin{block}{}
		in/out $\rightarrow$ en/el
	\end{block}
	
	\dots
\end{frame}  
    
  \subsection{My examples}
  
  \begin{frame}
  	\frametitle{User collection}

	\begin{block}{}
		users $\rightarrow$ uzantaro
	\end{block}
	\begin{block}{}
		? $\rightarrow$ uzantaroj
	\end{block}
	
  \end{frame}
  
  \begin{frame}
  	\frametitle{Part: ero}

	\begin{block}{}
		imagePart $\rightarrow$ bildero
	\end{block}
	\begin{block}{}
		requestPart $\rightarrow$ petero
	\end{block}
	
  \end{frame}
  
  \begin{frame}
  	\frametitle{Toolset}

	\begin{block}{}
		Toolset $\rightarrow$ ilaro
	\end{block}
	\begin{block}{}
		Trello toolset $\rightarrow$ trelloilaro
	\end{block}
  \end{frame}
  
  \begin{frame}
  	\frametitle{Bigger examples}
  	\begin{block}{}
  		\lstinputlisting[firstline=14, lastline=18, basicstyle=\small]{ordonoj.scala}
	\end{block}
  \end{frame}
    
  \begin{frame}
  	\frametitle{Some sparkles for the discussion\dots ?}
  	Do you think it would be worth to design some perfect language to be used by programmers over the world, equiped to well reflect the needs (based on esperanto, english or elvish -- whatever), let's just assume it would be perfect
  	\begin{itemize}
  		\item opracowanoby nowy, lepszy niż esperanto język sztuczny do używania przez programistów,
  		\item zostałby natychmiast wprowadzony do nauki w uniwersytetach oraz uzyskał wsparcie biznesu
  	\end{itemize}
  	
  	z perspektywy ludzkości -- czy ,,inwestycja'' zwróciłaby~się? Po jakim czasie?
  \end{frame}

\end{document}