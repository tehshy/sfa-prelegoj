\documentclass{beamer}
\usepackage[T1]{fontenc}
\usepackage[utf8x]{inputenc} 
\usepackage[esperanto,polish]{babel}
%\usepackage{umbcboxes}
\usepackage{listings}

\title[Esperanto w programowaniu]{Zalety używania planowego języka aglutacyjnego do programowania na przykładzie esperanta.}
%\subtitle{}
\author{Tomasz Szymula}
\institute[PEJ]{Koło naukowe Blabel, AGH}

\date[2014]{51-wsza sesja kół naukowych,\\ \today}
\subject{Lingvistiko}
%Zalety używania planowego języka aglutacyjnego do programowania na przykładzie esperanta.

%Język angielski jest standardem, jeśli chodzi o język w którym pisany
%jest zarówno kod (np identyfikatory) jak i komentarze. Zastosowanie do
%tego celu języka z regularnym słowotwórstwem opartym o sklejanie
%zrostków daje interesujące rezultaty. Omówienie kilku przydatnych
%afiksów esperanta z przykładami oraz porównaniem do języka angielskiego
\usetheme{Antibes}
\usecolortheme{dolphin}

\AtBeginSubsection[]
{
  \begin{frame}
    \frametitle{Plan}
    \tableofcontents[currentsubsection]
  \end{frame}
}
% "define" Scala
\lstdefinelanguage{Scala}{
  morekeywords={abstract,case,catch,class,def,%
    do,else,extends,false,final,finally,%
    for,if,implicit,import,match,mixin,%
    new,null,object,override,package,%
    private,protected,requires,return,sealed,%
    super,this,throw,trait,true,try,%
    type,val,var,while,with,yield},
  otherkeywords={=>,<-,<\%,<:,>:,\#,@},
  sensitive=true,
  morecomment=[l]{//},
  morecomment=[n]{/*}{*/},
  morestring=[b]",
  morestring=[b]',
  morestring=[b]"""
}

\lstset{language=Scala} 

\begin{document}
  \frame{\titlepage}
 
  \section{Esperanto}
  \subsection{Czym jest} 
  
  \begin{frame}
  	\frametitle{Esperanto}
  	
  	\begin{block}{Esperanto to język}
  		\begin{itemize}
  			\item planowy,
			\item międzynarodowy,
			\item pomocniczy.
		\end{itemize}
  	\end{block}
  	
  	\pause

\definecolor{antiquefuchsia}{rgb}{0.57, 0.36, 0.51}
\definecolor{ao(english)}{rgb}{0.0, 0.5, 0.0}
\definecolor{blue(pigment)}{rgb}{0.2, 0.2, 0.6}
	\begin{block}{A dla lingwisty aglutynacyjny}
		Na przykład, bo lubię czytać\dots \textbf{\textcolor{antiquefuchsia}{libr}o\textcolor{ao(english)}{j}\textcolor{blue(pigment)}{n}}
	\end{block}  	
  	
  \end{frame}
 
  
  
  \subsection{Jakie jest}
  
  \begin{frame}
  	\frametitle{Rdzenie}
  	\framesubtitle{Za broszurą ,,Malkovru Esperanton''}
  	
  	\begin{block}{Słowotwórstwo}
  		Oparte o system prefiksów i sufiksów
  	\end{block}
	Rdzenie wyrazów pochodzą z języków:
	\begin{itemize}
		\item ~75\% z romańskich,
		\item ~20\% z germańskich,
		\item ~5\% z innych.
	\end{itemize}
		
  \end{frame}
  
%**************************************** IL 
  \begin{frame}
  	\frametitle{Afiksy}
  	\framesubtitle{-il}
  	  	
  		\begin{block}{-il}
  			\begin{center}
  				narzędzie
  			\end{center}
  		\end{block}
  	
  		\begin{itemize}
  			\item<1-> tranĉi -- kroić $\rightarrow$ tranĉilo -- nóż
%  			\item<1-> kompili -- kompilować $\rightarrow$ kompililo -- kompilator
  			\item<1-> ligi -- łączyć $\rightarrow$ ligilo -- link
%  			\item<2-> manĝi -- jeść $\rightarrow$ \pause manĝilo -- sztuciec
  		\end{itemize}
  \end{frame}

%**************************************** -AR
  \begin{frame}
  	\frametitle{Afiksy}
  	\framesubtitle{-ar}
  	  	
  		\begin{block}{-ar}
  			\begin{center}
  				zbiór
  			\end{center}
  		\end{block}
  	
  		\begin{itemize}
  			\item<1-> arbo -- drzewo $\rightarrow$ arbaro -- las
%  			\item<1-> homo -- człowiek $\rightarrow$ homaro -- ludzkość
  			\item<1-> klavo -- klawisz $\rightarrow$ klavaro -- klawiatura
%  			\item<2-> manĝilo - sztuciec $\rightarrow$ \pause manĝilaro - zastawa
  		\end{itemize}
  \end{frame}

%**************************************** -ER
  \begin{frame}
  	\frametitle{Afiksy}
  	\framesubtitle{-er}
  	  	
  		\begin{block}{-er}
  			\begin{center}
  			cząstka
  			\end{center}
  		\end{block}
  	
  		\begin{itemize}
			\item<1-> programo -- program $\rightarrow$ programero -- punkt programu
			\item<1-> blogo -- blog $\rightarrow$ blogero -- wpis na blogu
			
  		\end{itemize}
  \end{frame}

%  \begin{frame}
%  	\frametitle{Ogólny przykład do afiksów}
%  	
%  	coś w stylu:\\
%  	manĝi - jesc,\\
%  	manĝilo - sztuciec\\
%  	manĝilaro - zastawa\\
%  	manĝeti - podjadac\\
%  	manĝegi - "wcinac"\\
%  	antaŭmanĝi - przedjadac (?)\\
%  	itd.
%
%  \end{frame}
 
  \section{W praktyce}
  \subsection{Wzorce projektowe}
  
\begin{frame}
	\frametitle{Wybrane wzorce projektowe}
	
	\begin{block}{}
		Observer, Observable $\rightarrow$ Spektanto, Spektebla
	\end{block}
	
	\begin{block}{}
		Decorator $\rightarrow$ Dekoranto
	\end{block}
	
	\begin{block}{}
		Interpreter $\rightarrow$ Interpretilo
	\end{block}
	
	\dots
\end{frame}  

\begin{frame}
	\frametitle{Wejście/wyjście}
	
	\begin{block}{}
		in/out $\rightarrow$ en/el
	\end{block}
	
	\dots
\end{frame}  
    
  \subsection{Moje przykłady}
  
  \begin{frame}
  	\frametitle{Kolekcja użytkowników}

	\begin{block}{}
		users $\rightarrow$ uzantaro
	\end{block}
	\begin{block}{}
		? $\rightarrow$ uzantaroj
	\end{block}
	
  \end{frame}
  
  \begin{frame}
  	\frametitle{Część: ero}

	\begin{block}{}
		imagePart $\rightarrow$ bildero
	\end{block}
	\begin{block}{}
		requestPart $\rightarrow$ petero
	\end{block}
	
  \end{frame}
  
  \begin{frame}
  	\frametitle{Zbiór narzędzi}

	\begin{block}{}
		Toolset $\rightarrow$ ilaro
	\end{block}
	\begin{block}{}
		Trello toolset $\rightarrow$ trelloilaro
	\end{block}
  \end{frame}
  \begin{frame}
  	\frametitle{Wiekszy przykład}
  	\begin{block}{}
  		\lstinputlisting[firstline=14, lastline=18, basicstyle=\small]{ordonoj.scala}
	\end{block}
  \end{frame}
    
  \begin{frame}
  	\frametitle{Podsumowanie}
  	Jeśli 
  	\begin{itemize}
  		\item opracowanoby nowy, lepszy niż esperanto język sztuczny do używania przez programistów,
  		\item zostałby natychmiast wprowadzony do nauki w uniwersytetach oraz uzyskał wsparcie biznesu
  	\end{itemize}
  	
  	z perspektywy ludzkości -- czy ,,inwestycja'' zwróciłaby~się? Po jakim czasie?
  \end{frame}

\end{document}